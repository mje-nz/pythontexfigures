\documentclass[12pt,a4paper]{article}
\usepackage[utf8]{inputenc}

\usepackage{pgf}
\usepackage{subfig}

% Make build reproducible
% https://tex.stackexchange.com/a/5961
% https://tex.stackexchange.com/a/313605
\pdfinfo{ /Producer (pdfTeX) }
\pdfinfoomitdate=1
\pdftrailerid{}
\pdfsuppressptexinfo=-1

% gobble=auto lets you indent inside code environments
% keeptemps=all keeps the code and output files so you can see what executed
\usepackage[gobble=auto,keeptemps=all]{pythontex}
\usepackage{pythontexfigures}
\pythontexfigurespath{scripts}

\begin{pythontexcustomcode}[begin]{py}
    # Set up environment for scripts
    import numpy as np
    import matplotlib.pyplot as plt
    import seaborn as sns
    sns.set_style("white")
    import pythontexfigures as ptf
\end{pythontexcustomcode}


\begin{document}
    \title{Latex Python figure examples}
    \author{Matthew Edwards}
    \date{July 2019}
    \maketitle

    \begin{figure}[h]
        \pyfig{"test.py"}
        \caption{Test Python figure.}
        \label{fig:test}
    \end{figure}

    \begin{figure}
        \centering
        \subfloat[Test figure (golden aspect ratio).]{%
        \pyfig{"test.py", width=0.45*ptf.TEXT_WIDTH, aspect=ptf.GOLDEN}
        }
        \subfloat[Test figure (square aspect ratio).]{%
        \pyfig{"test.py", width=0.45*ptf.TEXT_WIDTH}
        }
        \caption{Side-by-side Python figures.}
        \label{fig:test2}
    \end{figure}
\end{document}
