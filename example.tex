\documentclass[12pt,a4paper]{article}
\usepackage[utf8]{inputenc}

\usepackage{graphicx}
\graphicspath{{./img/}}
\usepackage{pgf}
\usepackage{subfig}

\usepackage{pythontex}
\setpythontexcontext{textwidth=\the\textwidth}
\begin{pythontexcustomcode}[begin]{py}
FIGURES_DIR = 'figures'
DATA_DIR = 'data'
SCRIPTS_DIR = 'scripts'

import sys
sys.path.append(SCRIPTS_DIR)

pytex.add_dependencies('python_figures.py')
with pytex.open('python_figures.py', 'rb') as file:
    exec(compile(file.read(), filename='python_figures.py', mode='exec'), globals())
\end{pythontexcustomcode}

\begin{document}
	\title{Latex Python figure example}
	\author{Matthew Edwards}
	\date{July 2019}
	\maketitle

	Test: \py{1 + 1}. Text width: \py{pytex.context.textwidth}

	\begin{figure}
		\py{python_fig('test.py')}
		\caption{Test figure.}
		\label{fig:test}
	\end{figure}

	\begin{figure}
		\centering
		\subfloat[Test figure.]{%
			\py{python_fig('test.py', width=0.45*textwidth())}
		}
		\subfloat[Test figure.]{%
			\py{python_fig('test.py', width=0.45*textwidth())}
		}
		\caption{Side-by-side Python figures.}
		\label{fig:test2}
	\end{figure}

	TODO: PythonTeX doesn't re-run when python files change.
\end{document}
