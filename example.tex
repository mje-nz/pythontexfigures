\documentclass[12pt,a4paper]{article}
\usepackage[utf8]{inputenc}

\usepackage{graphicx}
\graphicspath{{./img/}}
\usepackage{pgf}
\usepackage{subfig}

\usepackage{pythontex}
\setpythontexcontext{textwidth=\the\textwidth}
\begin{pythontexcustomcode}[begin]{py}
FIGURES_DIR = 'figures'
DATA_DIR = 'data'
SCRIPTS_DIR = 'scripts'

with pytex.open('python_figures.py', 'rb') as file:
    exec(compile(file.read(), filename='python_figures.py', mode='exec'), globals())
\end{pythontexcustomcode}
\newcommand{\pyfig}[1]{\py{python_fig(#1)}}

\begin{document}
	\title{Latex Python figure examples}
	\author{Matthew Edwards}
	\date{July 2019}
	\maketitle

	\begin{figure}[h]
		\pyfig{'test.py'}
		\caption{Test Python figure.}
		\label{fig:test}
	\end{figure}

	\begin{figure}
		\centering
		\subfloat[Test figure (golden aspect ratio).]{%
			\pyfig{'test.py', width=0.45*textwidth(), aspect=GOLDEN}
		}
		\subfloat[Test figure (square aspect ratio).]{%
			\pyfig{'test.py', width=0.45*textwidth()}
		}
		\caption{Side-by-side Python figures.}
		\label{fig:test2}
	\end{figure}
\end{document}
